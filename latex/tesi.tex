\documentclass[12pt,a4paper]{report}
\usepackage[italian]{babel}
\usepackage{newlfont}
\usepackage{color}
\textwidth=450pt\oddsidemargin=0pt


\begin{document}
	\begin{titlepage}
		%
		%
		% UNA VOLTA FATTE LE DOVUTE MODIFICHE SOSTITUIRE "black" CON "BLACK" NEI COMANDI \textcolor
		%
		%
		\begin{center}
			{{\Large{\textsc{Alma Mater Studiorum $\cdot$ Universit\`a di Bologna}}}} 
			\rule[0.1cm]{15.8cm}{0.1mm}
			\rule[0.5cm]{15.8cm}{0.6mm}
			\\\vspace{3mm}
			
			{\small{\bf Dipartimento di Fisica e Astronomia “Augusto Righi”\\
					Corso di Laurea in Fisica}}
			
		\end{center}
		
		\vspace{23mm}
		
		\begin{center}\textcolor{black}{
				%
				% INSERIRE IL TITOLO DELLA TESI
				%
				{\LARGE{\bf Facility for $\lambda_c$ deuteron production in ALICE}}\\
		}\end{center}
		
		\vspace{50mm} \par \noindent
		
		\begin{minipage}[t]{0.47\textwidth}
			%
			% INSERIRE IL NOME DEL RELATORE CON IL RELATIVO TITOLO DI DOTTORE O PROFESSORE
			%
			{\large{\bf Relatore: \vspace{2mm}\\\textcolor{black}{
						Prof. Andrea Alici}\\\\
					%
					% INSERIRE IL NOME DEL CORRELATORE CON IL RELATIVO TITOLO DI DOTTORE O PROFESSORE
					%
					% SE NON AVETE UN CORRELATORE CANCELLATE LE PROSSIME 3 RIGHE
					%
					\textcolor{black}{
						\bf Correlatore: 
						\vspace{2mm}\\
							Dott. Nicolò Jacazio \\\\}}}
		\end{minipage}
		%
		\hfill
		%
		\begin{minipage}[t]{0.47\textwidth}\raggedleft \textcolor{black}{
				{\large{\bf Presentata da:
						\vspace{2mm}\\
						%
						% INSERIRE IL NOME DEL CANDIDATO
						%
						Giuseppe Luciano}}}
		\end{minipage}
		
		\vspace{40mm}
		
		\begin{center}
			%
			% INSERIRE L'ANNO ACCADEMICO
			%
			Anno Accademico \textcolor{black}{ 2024/2025}
		\end{center}
		
	\end{titlepage}
	\newpage
	
	\tableofcontents  % Questo comando genera l'indice
	
	\newpage
	
	\section{Abstract}
	Testo dell'introduzione...
	
	\newpage
	
	\section{Capitolo 1}
	Testo del capitolo 1...
	
	\newpage
	
	\subsection{Sezione 1.1}
	Testo della sezione 1.1...
	
	\newpage
	
	\subsection{Sezione 1.2}
	Testo della sezione 1.2...
	
	\newpage
	
	\section{Capitolo 2}
	Testo del capitolo 2...
	
	\newpage
	
	\subsection{Conclusion}
	Testo della sezione 2.1...
	
	\newpage
	
\end{document}
