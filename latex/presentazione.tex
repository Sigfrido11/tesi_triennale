\documentclass[11pt]{beamer}

\usetheme{Madrid} % Tema sobrio ed elegante
%\usecolortheme{Madrid} % Colorazione leggera
\usefonttheme{professionalfonts}

% Pacchetti utili
\usepackage[utf8]{inputenc}
\usepackage[italian]{babel}
\usepackage{graphicx}
\usepackage{lmodern}
\usepackage{hyperref}
\usepackage[table]{xcolor}
\usepackage[backend=biber, sorting=none]{biblatex}  
\addbibresource{citations.bib}

% Informazioni della tesi
\title[Facility for c-deuteron production in ALICE]{Facility for c-deuteron production in ALICE}
\author[Giuseppe Luciano]{\textbf{Giuseppe Luciano}}
\institute[Università di Bologna]{
	\normalsize{
		\textbf{Dipartimento di Fisica e Astronomia “Augusto Righi”\\
		} \\
		Corso di Laurea in Fisica
	}\\[1ex]
	\textbf{Relatore:} Prof. Andrea Alici \\
	\textbf{Correlatore:} Dott. Nicolò Jacazio
}
\date{19 Settembre 2025}

% Inizio documento
\begin{document}
	
	% Slide iniziale
	\begin{frame}
		\titlepage
	\end{frame}
	
	\begin{frame}
		\frametitle{C-deuteron}
		
		\begin{figure}		
			\centering
			\includegraphics[width=0.45 \linewidth]{pictures/c-deuteron.png}
			\caption{L'immagine mostra una rappresentazione schematica del c-deuteron}
		\end{figure}	
		\begin{itemize}
			\item Massa $\Lambda_c^+$: $m_{\Lambda_c^+} = (2286.46 \pm 0.14)$ MeV/$c^2$ \cite{ParticleDataGroup:2024cfk}.
			\item Massa neutrone: $m_N$ = 939.56542052(54) MeV/$c^2$ \cite{ParticleDataGroup:2024cfk}.
			\item Energia legame Deutone: $E_{be, d} = (2.224575 \pm 0.000009)\,\text{MeV}$
			\cite{VANDERLEUN1982261}.
			\item Massa c-deuteron: $m_{c-d} \approx 3.223$ GeV/$c^2$.
		\end{itemize}
		
	\end{frame}
	
	
	\begin{frame}
		\frametitle{Simulazione}
		Parametri della simulazione:
		\begin{itemize}
			\item Fireball a simmetria sferica.
			\item Numero eventi generati: N = $2 \cdot \ 10^7$.
			\item Ensemble gran-canonico per tutti gli adroni ad eccezione dei mesoni per i quali è stato utilizzato il formalismo basato sulla statistica di Bose-Einstein.
			\item Potenziale barionico: $\mu_B=0.71 \pm 0.45$ MeV \cite{charm_hierarchy_in_the_statistical_hadronization_model}.
			\item Potenziale della carica elettrica: $\mu_Q=-0.18 \pm 0.90$ MeV \cite{charm_hierarchy_in_the_statistical_hadronization_model}.
			\item Fugacità di charm $\gamma_c = 29.6 \pm 5.2$ \cite{charm_hierarchy_in_the_statistical_hadronization_model}.
			\item Temperatura di freeze-out: $T= 156$ MeV.
			\item Raggio di freeze-out: $r= 8$ fm.
		\end{itemize}
		
	\end{frame}
	
	
	\begin{frame}
		\frametitle{Molteplicità}
		\begin{figure}
			\includegraphics[width=0.9\linewidth]{pictures/first_graph.pdf}
			\caption{La figura mostra dN/dy di differenti adroni per $|y|<$0.5. Si sono utilizzate le unità naturali.}
		\end{figure}
		
	\end{frame}
	
	\begin{frame}
		\frametitle{Relazione funzionale}
		La relazione funzionale utilizzata per il fit è della forma $dN/dy= a e^{bx}$, con a e b parametri liberi. 
		
		Per i nuclei senza quark charm si è ottenuto:
		\begin{itemize}
			\item a = (1.057 $\pm$ 0.014) $\cdot \ 10^5$.
			\item b = (6.2107 $\pm$ 0.0071) $GeV^{-1}$. 
			\item $\tilde{\chi}^2 = 0.16$. 
		\end{itemize}
		Per i nuclei contenenti il quark charm invece è stato fissato il parametro b al valore precedentemente ottenuto per poi determinare a = (3.534 $\pm$ 0.021) $\cdot \ 10^5$.
		Nello specifico per il c-deuteron si è ottenuto $(dN/dy)_{c-d}= (7.072 \pm 0.042) \cdot 10^{-4}$.
	\end{frame}
	
	\begin{frame}
		\frametitle{Variazioni nel raggio di freeze-out}
		Dalla meccanica statistica è noto che:
		\begin{equation}
			\langle N \rangle = \frac{1}{\beta} \left( \frac{\partial \ln Z}{\partial \mu_B}\right)
			\label{eq:stat_n}
		\end{equation}
		
		e che:
		\begin{equation}
			\boxed{\ln Z_i(T,V,\mu_i) =
				\frac{\Delta V g_i}{2\pi^2 \hbar^3}
				\int_{0}^{\infty} p^2 \, dp \,
				\ln \left( 1 + \theta_i e^{\beta(\mu_i - E)} \right)}
			\label{eq:partition_function}
		\end{equation}
		
		Quindi:
		\begin{equation}
			\boxed{\langle N \rangle =
				\frac{\Delta V g_i}{2\pi^2 \hbar^3}
				\int_{0}^{\infty} dp \,
				\frac{p^2}{e^{-\beta(\mu_i - E)} + \theta_i}}
			\label{eq:mean_particle_number}
		\end{equation}
		
		Assumendo una simmetria sferica:
		\begin{equation}
			\langle N \rangle =
			\frac{\tfrac{4}{3}\pi r^3 g_i}{2\pi^2 \hbar^3}
			\int_{0}^{\infty} dp \,
			\frac{p^2}{e^{-\beta(\mu_i - E)} + \theta_i}
			\;\;\propto r^3
		\end{equation}
	\end{frame}
	
	\begin{frame}
		\frametitle{Variazioni nel raggio di freeze-out}
		\begin{figure}
			
			\centering
			\includegraphics[width=1\linewidth]{pictures/var_radius.pdf	}
			\caption{La figura mostra dN/dy del c-deuteron nel range $|y| < 0.5$ per le variazioni nel raggio di freeze-out.}
			
		\end{figure}
	\end{frame}
	
	\begin{frame}
		\frametitle{Variazioni nel raggio di freeze-out}
		Il fit è stato eseguito utilizzando la seguente relazione funzionale $dN/dy =a r^3 +b$, con a e b parametri da determinare. I risultati del fit sono stati:
		\begin{itemize}
			\item a = (1.3921 $\pm$ 0.0032) $\cdot \ 10^{-6} \ fm^{-3}$.
			\item b = (2.43 $\pm$ 0.13) $\cdot 	\ 10^{-8}$.
			\item $\tilde{\chi}^2$ = 0.42.
		\end{itemize}	
		
	\end{frame}
	
	
	\begin{frame}
		\frametitle{Variazioni nella temperatura di freeze-out}
		
		\begin{figure}
			\centering
			\includegraphics[width=1 \linewidth]{pictures/var_temperaure.pdf}
			\caption{La figura mostra dN/dy del c-deuteron nel range $|y| < 0.5$ per le variazioni nella temperatura di freeze-out.}
		\end{figure}
	\end{frame}
	
	
	\begin{frame}
		\frametitle{Variazioni con la temperatura}
		
		Il fit è stato eseguito utilizzando la seguente relazione funzionale 
		$dN/dy = a e^{bT} + c$, con a, b e c parametri da determinare. 
		I risultati del fit sono stati:
		
		\begin{itemize}
			\item a = (1.19 $\pm$ 0.17) $\cdot \ 10^{-12}$.
			\item b = (0.13009 $\pm$ 0.00087) $MeV^{-1}$.
			\item c = (-6.38 $\pm$ 0.50) $\cdot \ 10^{-6}$.
			\item $\tilde{\chi}^2$ = 1.0.
		\end{itemize}
		
		Quest'ultimo valore suggerisce che, sebbene non sia stato possibile procedere analiticamente, l'accordo fra dati simulati e la relazione empirica risulta comunque consistente.
		
	\end{frame}
	
	\begin{frame}
		\frametitle{Variazioni nella fugacità di charm}
		Utilizzando la precedente relazione è possibile osservare come:
		\begin{equation}
			\begin{aligned}
				\langle N \rangle &= \frac{\Delta V g_i}{2\pi^2\hbar^3} 
				\int_{0}^{\infty} dp 
				\frac{p^2}{e^{-\beta(\mu_i-E)}+\theta_i} \\
				&\approx \frac{\Delta V g_i}{2\pi^2\hbar^3} 
				\int_{0}^{\infty} dp \ p^2 e^{\beta(\mu_i-E)} \\
				&= \frac{\Delta V g_i}{2\pi^2\hbar^3} e^{\beta \mu_i} 
				\int_{0}^{\infty} dp \ p^2 e^{-\beta E} \\
				&= \frac{\Delta V g_i}{2\pi^2\hbar^3} \gamma_c 
				\int_{0}^{\infty} dp \ p^2 e^{-\beta E}\\
				&\propto \gamma_c
			\end{aligned}
		\end{equation} 
	\end{frame}
	
	
	\begin{frame}
		\frametitle{Variazioni nella fugacità di charm}
		\begin{figure}
			\centering
			\includegraphics[width=0.7 \linewidth]{pictures/var_fugacity.pdf}
			\caption{La figura mostra dN/dy del c-deuteron nel range $|y| < 0.5$ per le variazioni nella fugacità di charm.}
			\label{fig:var_fugacity}
		\end{figure}
	\end{frame}
	
	\begin{frame}
		\frametitle{Variazioni nella fugacità di charm}
		Il fit è stato eseguito utilizzando la seguente relazione funzionale $dN/dy = a x + b$, con $a$ e $b$ parametri da determinare. I risultati del fit sono stati:
		\begin{itemize}
			\item $a = (2.366 \pm 0.032) \cdot 10^{-5}$.
			\item $b = (2.53 \pm 0.94) \cdot 10^{-5}$.
			\item $\tilde{\chi}^2 = 1.4$.
		\end{itemize}
	\end{frame}
	
	\begin{frame}
		\frametitle{Decadimenti}
		\begin{table}[h]
			\centering
			\begin{tabular}{c|c}
				\hline
				\cellcolor{yellow} \text{Channel} & \cellcolor{yellow} \text{Branching ratio} \\
				\hline
				d $\bar{K^0}$ & (2.3 $\pm$ 0.6)\% x 7\% \\
				\hline
				d $K^- \ \pi^+$ &  (5.0 $\pm$ 1.3)\% x 7\% \\
				\hline
				d $\bar{K^*}(892)$ &  (1.6 $\pm$ 0.5)\% x 7\% \\
				\hline
				d $\bar{K^0} \ \pi^0$ &  (3.3 $\pm$ 1.0)\% x 7\% \\
				\hline
				d $\bar{K^0} \ \eta$ &  (1.2 $\pm$ 0.4)\% x 7\% \\
				\hline
				d $\bar{K^0} \ \pi^+ \ \pi^-$ &  (2.6 $\pm$ 0.7)\% x 7\% \\
				\hline
				d $K^- \ \pi^+ \ \pi^0$ &  (3.4 $\pm$ 1.0)\% x 7\% \\
				\hline
				totale & (1.36 $\pm$ 0.39)\% \\
				\hline
			\end{tabular}
			\caption{La tabella mostra tutti i canali di decadimento considerati per la presente analisi. La prima colonna mostra i possibili prodotti del decadimento mentre la seconda i relativi branching ratio con i relativi errori. Il termine x 7\% significa che il branching ratio deve essere ridotto per la probabilità di formare uno stato legato fra protone e neutrone. In questo caso d indica il deutone.}
			\label{tab:decay_channel}
		\end{table}
		
	\end{frame}
	
	\begin{frame}
		
		\begin{figure}		
			\centering
			\includegraphics[width=0.7\linewidth]{pictures/track_finding_efficiency_2.png}
			\caption{La figura mostra i grafici relativi all'efficienza di ricostruzione della traccia del Time Projection Chamber (TPC) di ALICE \cite{CERN-LHCC-2015-001}.}
		\end{figure}
	\end{frame}
	
	\begin{frame}
		\frametitle{Incremento deutoni}
		Sotto le seguenti ipotesi: 
		\begin{itemize}
			\item Il numero di collisioni effettuate dal LHC è $N_{LHC} = 10^{10}$;
			\item Sono state considerate solo le collisioni centrali, assunte pari al $5\%$ del totale;
			\item La temperatura di freeze-out è 156 MeV, con un raggio di freeze-out di 8 fm e una fugacità del charm pari a 29.6, assumendo simmetria sferica;
			\item Le abbondanze sono derivate unicamente dai canali di decadimento elencati nella tabella precedente;
			\item La probabilità di formazione di uno stato legato dopo il decadimento è del 7\%;
			\item Sono stati considerati solo i c-deuteron con $|y|<0.5$.
		\end{itemize}
		
	\end{frame}
	
	\begin{frame}
		Si è osservato come l'incremento del numero di deutoni la cui traccia viene ricostruita correttamente a causa del decadimento del c-deuteron è:
		\begin{equation}
			N_D = N_{LHC} \cdot p_D \cdot (dN/dy)_{c-d} \cdot \Delta y \cdot p_{central}
		\end{equation}
		Dove $N_{LHC} = 10^{10}$, $p_D$ è la probabilità che venga ricostruita la traccia di un deutone proveniente dal decadimento del c-deuteron $p_D = \frac{N_{detect}}{10^9}= (5.4823 \pm 0.0023) \ \cdot \ 10^{-3}$, $(dN/dy)_{c-d}= (7.072 \pm 0.042) \cdot 10^{-4}$. $\Delta y = 1$, $p_{central}$ è la probabilità che si verifichi una collisione centrale $p_{central} = 5\%$. 
		
		Ottenendo:
		\begin{center}
			$N_D$ = (1938 $\pm$ 12) deuterons.
		\end{center}
	\end{frame}
	
	\begin{frame}		
		\begin{figure}		
			\centering
			\includegraphics[width=0.5\linewidth]{pictures/ALICE_c-d_detection_2.png}
			\caption{La figura mostra la distribuzione della massa invariante, il fondo correlato e il fondo dovuto ai deuteroni primari.}
			\label{fig:QCD_vertices} 
		\end{figure}
	\end{frame}
	
	\begin{frame}
		\centering
		\Huge{\textbf{Grazie per l'attenzione.}}\\[2ex]
	\end{frame}
	
	\begin{frame}
		\frametitle{Bibliografia}
		\printbibliography
	\end{frame}
	
	% Slide finale
	
	
\end{document}

